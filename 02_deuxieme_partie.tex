% Titre de la partie
\section[Relativité restreinte]{Le point de vue de la Relativité Restreinte}

%%%%%%%%%%%%%%%%%%%%%%%%%%%%%%%%%%%%%%%%%%%%%%%%
% Première diapo (avec des équations)
%%%%%%%%%%%%%%%%%%%%%%%%%%%%%%%%%%%%%%%%%%%%%%%%
\begin{frame}
	\frametitle{Relativité Restreinte}
	\framesubtitle{Dynamique relativiste}

	\begin{block}{Nouvelle définition de la quantité de mouvement}
        \pause
		$$
		\vec{p} = \gamma\,m\vec{v}
		\qquad\text{avec}\qquad
		\gamma = \frac{1}{\sqrt{1 - \frac{v^2}{c^2}}}
		$$
	\end{block}

    \pause
		Avec des développements classique et
		en posant $u=1/r$, on en arrive à l'équation
		\onslide <4->{
		$$
		\onslide <5->
		\underbrace{
		\onslide <4->
		    \frac{\dd^2 u}{\dd \theta^2} + u = \frac{G M m\, E}{L^2\, c^2}
		\onslide <5->
		    }_{
		    \text{Partie usuelle}}
		\onslide <4->
		            +
		\onslide <6->
		\underbrace{
		\onslide <4->
		            \frac{\pa{G M m}^2}{L^2\, c^2}\, u
	    \onslide <6->
	               }_{\text{Partie relativiste}}
		$$
		}

\end{frame}


%%%%%%%%%%%%%%%%%%%%%%%%%%%%%%%%%%%%%%%%%%%%%%%%
% Deuxième diapo
%%%%%%%%%%%%%%%%%%%%%%%%%%%%%%%%%%%%%%%%%%%%%%%%
\begin{frame}
	\frametitle{Relativité Restreinte}
	\framesubtitle{Équation de l'ellipse}

	Équation différentielle remise en forme

    \pause
		$$
		    \frac{\dd u}{\dd \theta} + B^2\, u = A
            \pause
		        \quad \text{avec} \quad
		            B  = \sqrt{1 - \pa{\frac{G M m}{L\, c}}^{\!\!2} }
		$$

    \pause

	\begin{block}{Équation de l'\ofg{ellipse}}
		$$
		u = \frac{A}{B^2} \pa{1 + e\cos{\pac{B\pa{\theta - \theta_0}}}}
        \pause
		\quad\text{soit}\quad
		r = \frac{p}{1+e\cos{\pac{B\pa{\theta - \theta_0}}}}
		$$
	\end{block}

\end{frame}

%%%%%%%%%%%%%%%%%%%%%%%%%%%%%%%%%%%%%%%%%%%%%%%%
% Troisième diapo
%%%%%%%%%%%%%%%%%%%%%%%%%%%%%%%%%%%%%%%%%%%%%%%%
\begin{frame}
	\frametitle{Relativité Restreinte}
	\framesubtitle{Avance du périhélie}

	\begin{exampleblock}{Caractère non fermé}<+->
		Du fait que $B\neq1$.
	\end{exampleblock}

	\begin{exampleblock}{Rotation}<+->
		Entre deux périhélies successifs, $\theta$ tourne de $2\pi +\delta$ où
		\onslide<+->{
			$$
			\boxed{
		    \delta = 2\pi\pa{\frac{1}{B} - 1}
		    			\approx \pi \pa{\frac{G\, M\, m}{L\, c}}^{\!\!2}
		    			= \pi \, \frac{G M}{p\, c^2}
		    			= \pi\, \frac{G M}{a\, c^2\, \pa{1-e^2}}
		    }
			$$
		}
	\end{exampleblock}

	\begin{alertblock}{Malheureusement...}<+->
		l'application numérique ne donne \ofg{que}
		$7''$ d'arc par siècle...
	\end{alertblock}


\end{frame}
